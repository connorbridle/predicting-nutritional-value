\documentclass[11pt]{article}
\usepackage[margin=1in]{geometry}
\usepackage{natbib}
\bibliographystyle{agsm}
\begin{document}
\begin{titlepage}
	\begin{center}
	\line(1,0){400} \\
	[0.25in]
	\huge{\bfseries Third Year Project} \\
	[2mm]
	\line(1,0){400} \\
	[1.5cm]
	\textsc{\LARGE University of Keele}\\
	\textsc{\Large Department of Computer Science}\\
	[10cm]
	\end{center}
	\begin{flushright}
	\textsc{\large Connor Bridle. \\
	Computer science student\\
	\# 15018645 \\
	January 23, 2018 \\}
	\end{flushright}
\end{titlepage}
\tableofcontents 
\pagebreak
\section{Introduction}
The issue of healthy eating and obesity is widely researched and discussed within the Western World. Technology has often been attributed to the increase in worldwide obesity. In L.D. Rosen's et. al paper, using their predetermined categories, found that only 43\% of teenagers fell into the 'acceptable' range (REFERENCE nihms657969). However, with the increase in availability of technology and computing power; it can be used to modernise the fight against obesity. In this project, the aim is to use various computing techniques to analyse and classify whether or not certain food items are 'advisable' to eat. The outcome of this research will hopefully be an application that users can input nutrition information into and it will classify the food item.
	\subsection{General papers/fields of interest}
	
	\subsection{Key papers and literature}
	Stuff
	\subsection{Nudging}
	Nudging plays a pivotal role in improving the lives of the general populace; who often don't even know they are subjected to it. Having the ability to 'nudge' people to 	alter or change their lifestyle and behaviour is of great interest to governments 			around the world; including the UK government \citep{regulating2011judging}. Nudging is particularly relevant within this piece of research as influencing the 			behaviour of individuals, especially when it comes to dietary choices, is at the core 		of the project. Theresa Marteau and her colleagues put forward the point that people in general value their health yet persist in behaving in ways that undermine it \citep{regulating2011judging}. Common examples of nudging for weight loss include making salad the default order on a meal at a restaurant rather than chips. This subtle technique may make people more likely to order with a salad; therefore increasing the nutritional value of the meal.
	\subsection{Key nutrition/healthy eating standards literature}
	Stuff
	\subsection{Technologies role in promoting change}
	Stuff
	\subsection{Objectives of the project}
	The following are the key objectives and pieces of work to be produced throughout the process of the project.
	\begin{itemize}
		\item Identification of key nutritional measurements and the amount at which it is considered as high and low intake.
		\item Creation of a neural network to classify food items on whether or not is it 'advisable' to eat them.
		\begin{itemize}
			\item Creation of 3 datasets: A training dataset, a benchmark dataset and a testing dataset.
			\item Program the neural network implementation into the Java environment 
		\end{itemize}
		\item Creation of a piece of client software that will act as the interface to enter nutrition and additional information to the classification network.
		\begin{itemize}
			\item Creation of interface designs for the client-side application
			\item Implementing the design in the Java environment
		\end{itemize}
	\end{itemize}
\pagebreak

\section{Method}
	\subsection{The base classification problem}
		\subparagraph{Decision Tree algorithms}
		\subparagraph{Neural Networks}
		\subparagraph{Technique chosen for this project}
	\subsection{Design}
		\subparagraph{User Interface Design This is 
		the stuff to do with interface design}
		\subparagraph{Neural Network Design
			This is the stuff to do with NN design}
	\subsection{Implementation}
		\subparagraph{Implementing the client side application}
		\subparagraph{Implementing the back-end Neural Network}
	\subsection{Measurements}
		\subparagraph{Calories}
		\subparagraph{Fat}
		\subparagraph{Saturated Fat}
		\subparagraph{Carbohydrates}
		\subparagraph{Sugars}
		\subparagraph{Fibre}
		\subparagraph{Protein}
		\subparagraph{Salt}
		\subparagraph{Gender}
		\subparagraph{Age}
		\subparagraph{Time of day}
		\subparagraph{Level of activity}
	\subsection{Construction of dataset}
		\subparagraph{How is was constructed}
		\subparagraph{How many data samples}
\pagebreak

\section{Results}
Includes testing the system. What did you find? Often presented using figures, graphs, tables and screen-shots.
\pagebreak

\section{Discussion}
What did you find? What does it tell us? How does it relate to literature and your expectations? Critical evaluation of the results
\pagebreak

\section{Conclusion}
Includes further work. How could this work be developed, what were the shortcomings, why were certain objectives not achieved. How does this work contribute to the wider field.
	\subsection{Future work}
		\subparagraph{Applicability of QR codes placed on food items along with food 				labels}
		\subparagraph{Applicability of OCR to scan nutrition data from food labels}
		\subparagraph{Development of a mobile application}
		\subparagraph{Further investigation into the applicability of other types of neural networks; and testing whether these would be more efficient.}
	\subsection{Shortcomings of project}
		\subparagraph{The potential differences in cultures that could causes implications to the viability of the final solution.}
		\subparagraph{}
	\subsection{Objectives not achieved}
	Stuff
	$f(x)=1^4\cdot5.5$
\pagebreak
\bibliography{MyBib}
\end{document}