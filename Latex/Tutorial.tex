\documentclass[11pt]{article}

\begin{document}

\begin{titlepage}
	\begin{center}
	\line(1,0){300} \\
	[0.25in]
	\huge{\bfseries Third Year Project} \\
	[2mm]
	\line(1,0){300} \\
	[1.5cm]
	\textsc{\LARGE University of Keele}\\
	\textsc{\Large Department of Computer Science}\\
	[10cm]
	\end{center}
	\begin{flushright}
	\textsc{\large Connor Bridle. \\
	Computer science student\\
	\# 15018645 \\
	January 23, 2018 \\}
	\end{flushright}
\end{titlepage}
\section{Introduction}
This is the first section
	\subsection{General papers/fields of interest}
	Stuff
	\subsection{Key papers and literature}
	Stuff
	\subsection{Nudging}
	Nudging plays a pivotal role in improving the lives of the general populace; who often 	don't even know they are subjected to it. Having the ability to 'nudge' people to 			alter or change their lifestyle and behaviour is of great interest to governments 			around the world; including the UK government(REFERENCE 2 from Judging Nudging). 			Nudging is particularly relevant within this piece of research as influencing the 			behaviour of individuals, especially when it comes to dietary choices, is at the core 		of the project. Theresa Marteau and her colleagues put forward the point that people in general value their health yet persist in behaving in ways that undermine it (REFERENCE to Judging Nudging).
	\subsection{Key nutrition/healthy eating standards literature}
	Stuff
	\subsection{Technologies role in promoting change}
	Stuff
	\subsection{Objectives of the project}
	Stuff
\pagebreak

\section{Method}
	\subsection{Design}
		\subparagraph{User Interface Design This is 
		the stuff to do with interface design}
		\subparagraph{Neural Network Design
			This is the stuff to do with NN design}
	\subsection{Implementation}
		\subparagraph{Implementing the client side application}
		\subparagraph{Implementing the back-end Neural Network}
	\subsection{Measurements}
		\subparagraph{Calories}
		\subparagraph{Fat}
		\subparagraph{Saturated Fat}
		\subparagraph{Carbohydrates}
		\subparagraph{Sugars}
		\subparagraph{Fibre}
		\subparagraph{Protein}
		\subparagraph{Salt}
		\subparagraph{Gender}
		\subparagraph{Age}
		\subparagraph{Time of day}
		\subparagraph{Level of activity}
	\subsection{Construction of dataset}
		\subparagraph{How is was constructed}
		\subparagraph{How many data samples}
\pagebreak

\section{Results}
Includes testing the system. What did you find? Often presented using figures, graphs, tables and screen-shots.
\pagebreak

\section{Discussion}
What did you find? What does it tell us? How does it relate to literature and your expectations? Critical evaluation of the results
\pagebreak

\section{Conclusion}
Includes further work. How could this work be developed, what were the shortcomings, why were certain objectives not achieved. How does this work contribute to the wider field.
	\subsection{Future work}
		\subparagraph{Applicability of QR codes placed on food items along with food 				labels}
		\subparagraph{Applicability of OCR to scan nutrition data from food labels}
		\subparagraph{Development of a mobile application}
	\subsection{Shortcomings of project}
	Stuff	
	\subsection{Objectives not achieved}
	Stuff
	$f(x)=1^4\cdot5.5$
\pagebreak

\end{document}